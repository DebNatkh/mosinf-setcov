\begin{problem}{Ночная фотография}{Входные данные}{Результат}{}

Черно-белая цифровая фотография представляет собой таблицу из чисел размером $N\times N$, каждое число в которой кодирует яркость пикселя.

Для повышения качества черно-белой в съемки в темноте можно заменить яркость каждого пикселя на сумму яркостей в нем и в 8 соседних (по стороне или углу) пикселей. Для угловых пикселей и пикселей лежащих на границе считается только сумма яркостей пикселей, попадающих в фотографию.

В результате применения этой операции изображение <<замыливается>>, что не всегда хорошо. Вам необходимо по таблице, в которой посчитана сумма яркостей, восстановить исходную таблицу.

В первой строке записано число $N$~--- размер фотографии. В следующих $N$ строках задано по $N$ целых чисел: яркости пикселей после применения операции.

Вам необходимо вывести $N$ строк по $N$ чисел в каждой~--- яркости пикселей до применения операции. Если ответов несколько~--- выведите любой из них

В первом тесте $N = 5$. Оценка за этот тест: 30 баллов. Оценка за тест выставляется только в случае, если задание выполнено полностью правильно. Проверка осуществляется в режиме online (результат виден сразу).

Во втором тесте $N = 100$. Оценка за этот тест: 70 баллов. Оценка за тест выставляется только в случае, если задание выполнено полностью правильно. Во время тура проверяется. что сданный файл содержит 10000 чисел. Проверка правильности ответа осуществляется в режиме offline (результат виден после окончания тура).

\Examples
\begin{example}
\exmp{
3
12	21	16
27	45	33
24	39	28
}{1	2	3
4	5	6
7	8	9
}%
\end{example}

\end{problem} 